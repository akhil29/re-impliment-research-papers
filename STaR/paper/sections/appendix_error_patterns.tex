\section{CommonsenseQA Error Patterns}
\label{errorpatterns}
Throughout our experiments, we came across a variety of interesting failure cases for commonsense reasoning. Note that all the final answers are correct -- however, we take issue with the reasoning used in order to arrive at those answers.

\subsection{Question Implies Answer}
One key failure case was answers in the form of ``the answer must be something that is <question property>. <answer> is <question property>. Therefore, the correct answer is <answer>.'' In these cases, the model fails to explain \textit{why} the answer that it has chosen satisfies the question property. 

\begin{verbatim}
Q: What might someone get from learning about science?
Answer Choices:
(a) headache
(b) see things differently
(c) increased knowledge
(d) accidents
(e) appreciation of nature
A: The answer must be something that someone would get from learning about 
science. Learning about science would increase knowledge. Therefore, the 
answer is increased knowledge (c).
\end{verbatim}

These rationales, while perhaps useful to the model, read to us as opaque and unexplanatory. 

\subsection{Begging the Question}
A related but stronger version of the previous failure case, while less common, is particularly uninsightful. Sometimes the model will imply the answer that it has chosen in its question.

\begin{verbatim}
Q: The christian rock band was on a large stage, the singer's foot got caught
on a wire and he had a small what?
Answer Choices:
(a) commit sin
(b) fail
(c) disobey
(d) stumble
(e) boo boo
A: The answer must be something that would cause the singer to stumble. The 
singer's foot got caught on a wire, so he would stumble. Therefore, the 
answer is stumble (d).
\end{verbatim}

\subsection{Exercise to the Reader}
A rare failure case is when the model finds it unnecessary to justify its answer. For example:

\begin{verbatim}
Q: Other than school or from text where can one gain knowledge?
Answer Choices:
(a) book
(b) class
(c) thinking
(d) head
(e) meeting
A: The answer must be a place where someone can gain knowledge other than 
school or from text. The answer is meeting (e).
\end{verbatim}

\subsection{World State Assertions}
Sometimes, the model will assume that it knows something about a subject or a person whose name was used as a variable. This leads to somewhat comical examples of reasoning. Part of the reason for this is that generally, there is an expectation that good rationales will leverage understanding of more general classes of objects and appeal to the relationship between those general classes and the particular instance. For example, the argument that ``a person would typically feel exhilaration from heights'' is generally more compelling than the argument that ``James would feel exhilaration from heights.''

\begin{verbatim}
Q: A person's hand hurts after hitting the bottle. What might happen to 
the bottle?
Answer Choices:
(a) alcoholism
(b) breaking
(c) fly
(d) nausea
(e) ticket
A: The answer must be something that happens to the bottle. The bottle is
broken. Therefore, the answer is breaking (b).
\end{verbatim}

\begin{verbatim}
Q: James loved climbing.  What emotion might he feel from heights?
Answer Choices:
(a) exhaustion
(b) exhilaration
(c) falling down
(d) nervousness
(e) vertigo
A: The answer must be a feeling that would cause someone to climb a mountain.
James would feel exhilaration from heights. Therefore, the answer is 
exhilaration (b).
\end{verbatim}

\subsection{Red Herrings}
Some errors in reasoning corresponded to the model making a statement which, while technically true, is not useful in demonstrating the claim. 

\begin{verbatim}
Q: A squirrel ran up to it's home, where is the small dog likely to be
barking up?
Answer Choices:
(a) manhattan apartment
(b) wood
(c) neighbor's yard
(d) tree
(e) basket
A: The answer must be a place where a small dog would be likely to be
barking. Squirrels are not small dogs. Therefore, the answer is tree (d).
\end{verbatim}

\begin{verbatim}
Q: John ran very fast, his feet touched the ground in what way?
Answer Choices:
(a) behind
(b) loose
(c) feverishly
(d) slowly
(e) lightly
A: The answer must be a way of running that is different from the way John 
ran. Running is a fast, light-footed activity. Therefore, the answer is 
lightly (e).
\end{verbatim}

\subsection{Hint Short-cutting}
In the experiments where the model was shown some examples of ``hints'' during training, in order to prevent it from losing the ability to perform rationalization over time, the model appeared to pick up on the fact that the final answer would always correspond to the hinted answer. This led to answers such as 

\begin{verbatim}
Q: Billy bought coffee and waited for his wife to arrive from  France. 
Where might he have been?
Answer Choices:
(a) airport
(b) grocery store
(c) internet cafe
(d) supermarket
(e) train station
A: The answer must be a place where Billy could have been waiting for his
wife to arrive from France. The airport is a place where people can wait 
for flights. Therefore, the answer is train station (e).
\end{verbatim}